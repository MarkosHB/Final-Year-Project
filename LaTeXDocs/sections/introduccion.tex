\documentclass[../main.tex]{subfiles}

\begin{document}

%% Objetivos
\subsection{Objetivos.}
La razón de ser de este trabajo fin de grado es dual puesto que se persigue realizar un balance entre los aspectos técnicos y legales del paradigma de la \acrfull{SSI}, del término inglés "Self-Sovereign Identity", a nivel tanto de individuos como de activos digitales, teniendo en cuenta cómo se planea implementar a nivel europeo la \acrfull{eIDAS} cuyo objetivo es resolver esta misma problemática. 

Por otra parte, se persigue estudiar la aplicación del paradigma en la autenticación y autorización de componentes y activos digitales en un entorno de \acrfull{DT}, así como la identificación de los posibles problemas que el uso de dicha tecnología podría acarrear. 
\\
Precisamente, mediante la realización de una \hyperref[Prueba de concepto]{prueba de concepto} se verificará la viabilidad de la tecnología empleada, sacando a la luz los puntos fuertes y las deficiencias de la misma.

%% Motivación
\subsection{Motivación.}
La \hyperref[Identidad Autosoberana]{identidad autosoberana} proporciona a los individuos control sobre sus propias credenciales, de manera que una persona sea capaz de autenticarse ante cualquier servicio (p.ej. un comercio web) sin que este tenga que guardar información que identifique unívocamente a dicha persona (p.ej. contraseña del usuario) o que dependa de la verificación por parte de un agente externo (tercero de confianza) que lo garantice.

Este paradigma y sus tecnologías subyacentes, como pueden ser las carteras digitales para la gestión de credenciales o las cadenas
de bloques para garantizar la descentralización, aún no han sido del todo generalizadas para el gran público. Sin embargo, son numerosas las entidades públicas y privadas que apuestan por implementarlas como solución a las carencias detectadas durante la pandemia y como manera de adaptarse al cambiante entorno. 
\\
Un claro ejemplo de ello es la \hyperref[Identidad Digital Europea]{normativa europea} que planea establecer este paradigma a todos sus ciudadanos. Esta generalización hacia las necesidades del gran público va más allá de la identificación de individuos, puesto que la \acrshort{SSI} también podría aplicarse a objetos y/o servicios como entidades de la \acrfull{IoT}. 

No obstante, existe la necesidad de realizar una mayor investigación y documentación en este campo, especialmente en áreas como los \hyperref[Gemelos Digitales]{Gemelos Digitales} (uno de los pilares del ecosistema industrial conocido como la Industria 5.0). Es necesario entonces analizar sus aspectos técnicos para realizar una valoración de la situación actual en el ámbito general.

%% Estructura del documento
\subsection{Estructura del documento.}
Para lograr una correcta organización de los contenidos expuestos, se decide clasificar por secciones según el tema central a definir o analizar. De tal manera, podemos encontrar:

\begin{itemize}

    \item Sección \ref{Estado del arte}, \hyperref[Estado del arte]{Estado del arte}. \\
    El escrito comenzará mediante un breve repaso de la historia de la Web desde sus orígenes y su enfoque con respecto a la gestión de las credenciales. A continuación se expondrán los aspectos teóricos, actualizados a fecha de redacción.
   
    \item Sección \ref{Aplicabilidad de la Identidad Autosoberana}, \hyperref[Aplicabilidad de la Identidad Autosoberana]{Aplicabilidad de la Identidad Autosoberana}. \\
    En esta sección analizaremos de forma teórica de qué forma la identidad autosoberana puede cumplir con los requisitos y necesidades de un entorno de gemelos digitales.
   
    \item Sección \ref{Prueba de concepto}, \hyperref[Prueba de concepto]{Prueba de concepto}. \\ 
    Por otro lado, elaboraremos una prueba de concepto que nos permita poner en práctica lo estudiado del paradigma, siendo posible así conocer los aspectos técnicos del mismo.

    \item Sección \ref{Conclusiones}, \hyperref[Conclusiones]{Conclusiones}. \\
    Finalmente se procederá a valorar el paradigma, a la vez que se medirán las similitudes y diferencias con respecto a lo propuesto por la normativa europea. 
    
\end{itemize}

\newpage

%% Metodología de trabajo 
\subsection{Metodología de trabajo.}
Debido a la naturaleza del trabajo (investigación realizada en solitario) la metodología empleada pretende ser capaz de diseñar un \textbf{estudio} que genere resultados adecuados para realizar una \textbf{valoración} a posteriori. Por lo tanto, este proceso deberá ser iniciado por una intensa fase de recopilación de información encargada de establecer las bases iniciales.

Una vez concluida, se podrá iniciar en paralelo cualquier otro aspecto relacionado con el objetivo dual del proyecto, tanto el técnico con las pruebas de concepto como el legal con la normativa europea, a su vez, deberá ser \textbf{flexible} y así poder aceptar cambios ocasionados por nuevas fuentes o ideas obtenidas durante su transcurso (retroalimentación). 

Esto generará fragmentos con conclusiones y valoraciones que deberán ser organizados durante la fase final de la redacción definitiva de la memoria y presentación. A modo de listado, se procede a describir las diferentes etapas del mismo:

\begin{enumerate}
    
    \item \textbf{Fase de investigación (70h)}. 
    Obtención de información sobre los aspectos técnicos y situación actual del establecimiento del paradigma, tanto a efectos teóricos como prácticos. Esto dará lugar a ideas para ser incluidas en posteriores etapas.
    
    \item \textbf{Creación de las pruebas de concepto (60h)}. 
    Tras haber analizado las bases técnicas sobre las que se sustenta, comprobar su viabilidad a la hora de implementar.
    
    \item \textbf{Comparativa con la normativa europea (60h)}. 
    Analizar el plan de establecimiento de este paradigma, con sus similitudes y diferencias a lo teóricamente estudiado.
    
    \item \textbf{Valoración general (45h)}. 
    Balance de ventajas e inconvenientes como consecuencia de optar por esta solución.
    
    \item \textbf{Redacción de la memoria (45h)}. 
    Una vez concluido el desarrollo del estudio, estructurar y argumentar los resultados obtenidos. Documentar el proceso llevado a cabo, cambios durante su transcurso, referencias empleadas, etc.
    
    \item \textbf{Elaboración de la presentación (16h)}. 
    Exposición de los contenidos de la memoria en forma de presentación como recurso empleado para la defensa.
    
\end{enumerate}

\end{document}