\documentclass[../main.tex]{subfiles}

\begin{document}

% ----------------------
% Reflexiones personales
% ----------------------
\subsection{Reflexiones personales.}\label{Reflexiones personales}
Debido al objetivo de este \acrshort{TFG}, resultaría apropiado diferenciar las impresiones según el contexto en el que se han obtenido. Por lo tanto, hay que distinguir entre reflexiones de los mecanismos técnicos del paradigma y de su plan para la adopción.
\\


% --------------------------------------
% Sobre el paradigma de la \acrlong{SSI}
% --------------------------------------
\subsubsection{Sobre el paradigma de la \acrlong{SSI}.}
Gracias al estudio realizado, el conocimiento de causa adquirido permite desarrollar una opinión propia ante esta tecnología ya que tener preconcebida una idea sobre la misma suele ser un craso error, especialmente si somos escépticos ante la digitalización de aspectos críticos y sensibles como es la identidad propia.  
\\

\begin{enumerate}
    %----------------
    \item \underline{Solución como alternativa a la centralización.}
    \\ Pensar que este paradigma es perfecto y no existen los inconvenientes es una simpleza. De hecho, seguramente sea más sencillo y seguro si mantenemos el modelo centralizado tradicional para el almacenamiento de las credenciales, pero nunca arreglaríamos la gran dependencia a determinados proveedores de servicios que tenemos.

    Si hay algo que ha demostrado el paradigma es la fortaleza y robustez técnica suficiente para ser una alternativa viable con la que poder identificarnos digitalmente de manera descentralizada e independiente (que no es poco). No obstante tiene aspectos en los que mejorar, como solventar los problemas de escalabilidad o de recuperación ante pérdida de las credenciales, tal y como hemos tratado en el escrito.
    \\
    
    %----------------
    \item \underline{Críticas al diseño de las funciones de los actores.}
    \\ Tal y como se comentó en la \textit{Observación \ref{observacion-actores}}, el Titular no puede formar parte del sistema si no es mediante la interacción del Emisor con el \acrshort{VDR}. Desde mi punto de vista, este hecho cuya intención es ofrecer protección ante potenciales interacciones maliciosas, limita la autonomía y libertad del futuro poseedor de las \acrshort{VC}s en el paradigma. 
    
    Pongamos el caso en el que Emisor considera oportuno no atender la petición del Titular (por cualquier motivo, ya sea legal o por intereses personales). Diríamos entonces que el Titular se encuentra `vetado' de participar en el sistema aunque él quisiera ser parte.
    \\

\end{enumerate}


% ----------------------------------------------
% Sobre el establecimiento de la \acrlong{eIDAS}
% ----------------------------------------------
\subsubsection{Sobre el establecimiento de la \acrlong{eIDAS}.}
En mi opinión, no considero que los mecanismos técnicos del paradigma conlleven algún tipo de retroceso en los derechos o libertades del usuario (más bien es todo lo contrario). Sin embargo, el \hyperref[Balance general y tabla resumen]{balance general} hace visible que algunas de las principales vulnerabilidades surgen tras una incorrecta implementación de algún componente. Entonces me pregunto, ¿podemos fiarnos como ciudadanos de esta solución para gestionar nuestra identidad digital?
\\

\begin{enumerate}
    % ---------------
    \item \underline{Dificultades legales para su implantación.}
    \\ Actualmente existen dos protecciones (o escollos, según se mire) ante la implantación del paradigma para los ciudadanos europeos. Estos son el denominado como `Derecho al olvido' y el \acrfull{GDPR}.

    El primero hace referencia a la eliminación de determinada información personal en los medios digitales. No es de extrañar que el uso de Blockchain en los \acrshort{VDR}s pueda ir en contra de este principio por la propia naturaleza inmutable de dicha tecnología.

    El segundo choca directamente con la asociación de componentes técnicos como los \acrshort{DID}s y las \acrshort{VC}s con la persona (o entidad) física a la que representan. Aunque en un principio no pareciera ser un problema, ya que esa información la posee el Titular de las credenciales, sí lo es ante la pérdida o uso malicioso de estas.
    \\

    % ---------------
    \item \underline{Europa como líder regulador en tecnología.}
    \\ En los últimos años, la Comisión Europea ha destacado por llevar la iniciativa global en la regulación de nuevas tecnologías (Inteligencia Artificial) y la innovación en los métodos tradicionales de pago (Euro Digital) e identificación digital (\acrshort{eIDAS}). Personalmente, no descarto que se extrapolen estas medidas para abarcar el \acrshort{IoT} tras realizar un balance sobre los resultados obtenidos con estas regulaciones, debido a que a efectos prácticos la gestión de certificados del \acrshort{ITN} es idéntica que el propuesto por la \acrshort{EBSI} para las \acrshort{VC}s.

    Aunque esta decisión parece ser correcta, algunas voces destacan que puede llegar a perjudicar el correcto desarrollo de estas tecnologías. Además, no debe olvidarse que estas alternativas no deben bajo ningún concepto ser sustituto de los métodos físicos. 
    \\

\end{enumerate}

\begin{tcolorbox}[colback=gray!10!white, colframe=gray!50!black, title=Conclusión General]\label{conclusion-general}
Considero que es una mera cuestión de tiempo que adoptemos de forma voluntaria esta tecnología de la misma manera que ya lo hacemos con otras, como las transacciones bancarias inmediatas.
Para ello, Europa ejerce una excelente labor de tutelaje y protección durante la implantación del paradigma de la \acrshort{SSI}, ya sea mediante sus recursos legislativos o financieros. No obstante puede tornarse en una filtro innecesario el cual puede condicionar indirectamente numerosos aspectos de nuestro día a día.
\end{tcolorbox}

% --------------
% Líneas futuras
% --------------
\subsection{Líneas futuras.}\label{Líneas futuras}
En lo referente a trabajos continuistas a la labor desarrollada en este \acrshort{TFG}, existen determinados puntos que con gran probabilidad necesitarán actualizaciones recurrentes a medida que evolucionen los mecanismos técnicos del paradigma y se implanten al gran público.

Algunas ideas para futuros autores que mantengan la línea del estudio realizado son:
\begin{itemize}
    \item Desarrollar en profundidad las \textbf{implementaciones} existentes del paradigma.
    \item Expandir el análisis de \textbf{riesgos y amenazas} con situaciones no contempladas. 
    \item Incidir en otros \textbf{ámbitos} propensos a ser aplicados al paradigma, como el \acrshort{IoT}.
    \item Realizar un \textbf{seguimiento} al plan de implantación de la \acrshort{eIDAS} a los ciudadanos europeos.
\end{itemize}

\end{document}