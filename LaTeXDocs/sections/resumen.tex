\documentclass[../main.tex]{subfiles}

\begin{document}

%% begin abstract format
\makeatletter
\renewenvironment{abstract}{%
    \if@twocolumn
      \section*{Resumen \\}%
    \else %% <- here I've removed \small
    \begin{flushright}
        {\filleft\Huge\bfseries\fontsize{48pt}{12}\selectfont Resumen\vspace{\z@}}%  %% <- here I've added the format
        \end{flushright}
      \quotation
    \fi}
    {\if@twocolumn\else\endquotation\fi}
\makeatother
%% end abstract format
%% begin abstract format
\makeatletter
\renewenvironment{abstract}{%
    \if@twocolumn
      \section*{Resumen \\}%
    \else %% <- here I've removed \small
    \begin{flushright}
        {\filleft\Huge\bfseries\fontsize{48pt}{12}\selectfont Resumen\vspace{\z@}}%  %% <- here I've added the format
        \end{flushright}
      \quotation
    \fi}
    {\if@twocolumn\else\endquotation\fi}
\makeatother
%% end abstract format

% --------------
\begin{abstract}
La \acrlong{SSI} plantea una nueva alternativa a la gestión de las credenciales que permiten identificar a sus participantes en entornos digitales. Para ello se basa en determinados mecanismos técnicos que garantizan la seguridad criptográfica, en innovadoras tecnologías descentralizadoras fomentando así la transparencia en el sistema y en crear un ecosistema donde cada actor puede efectuar sus funciones de manera autónoma e independiente.
\\

Mediante este \acrshort{TFG} se pretende ofrecer unas correctas y actualizadas definiciones de dichos aspectos, las cuales servirán para elaborar un profundo análisis de aplicabilidad del paradigma en un entorno de \acrlong{DT}. Además, se introducirá el plan de progresiva implantación del mismo a la sociedad europea con la \acrlong{eIDAS}, se realizará una comparativa con otros entornos como el del \acrlong{IoT} y se desarrollará una prueba de concepto que ilustrará el funcionamiento de la tecnología desde un enfoque práctico.
\\

Finalmente, se presentarán a modo de conclusión un conjunto de reflexiones personales fruto de la labor de investigación, donde también se expondrán nuevas ideas y valoraciones que complementarán el análisis llevado a cabo.  
\\

\bfseries{\large{Listado de palabras clave:}} 
\begin{itemize}
    \item Identidad Autosoberana.
    \item Gemelos Digitales.
    \item Blockchain.
    \item Identidad Digital Europea.
    \item Internet de las Cosas.
\end{itemize}

\end{abstract}

\end{document}


