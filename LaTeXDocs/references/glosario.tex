\makeglossaries

% Estudio del arte
\newglossaryentry{hash}
{
    name=hash,
    description={Función criptográfica capaz de transformar una entrada de datos de cualquier tamaño en otra de longitud fija, la cual es única para la entrada proporcionada (por lo que pequeños cambios generan una salida completamente distinta) e irreversible (no se puede obtener la entrada a partir de la salida)}
}

\newglossaryentry{HTTPS}
{
    name=HTTPS,
    description={Protocolo construido sobre HTTP que introduce una capa de seguridad (SSL/TLS) que permite el intercambio de información entre dispositivos de manera segura}
}

% SSI
\newglossaryentry{criptodivisas}
{
    name=criptodivisas,
    description={Método de intercambio de monedas digitales que emplea criptografía para realizar intercambios seguros y estructuras de datos como Blockchain para garantizar la descentralización}
}

\newglossaryentry{URI}
{
    name=Identificador Uniforme de Recursos,
    description= {Un URI es el formato de identificador estándar para todos los recursos en la World Wide Web. Más información en \cite{DID-core}}
}

\newglossaryentry{firma digital}{
    name=firma digital,
    description= {Mecanismo criptográfico utilizado para garantizar la autenticidad e integridad de un mensaje, documento o cualquier otro tipo de dato digital}
}

% Europa
\newglossaryentry{Github}{
    name=Github,
    description= {Principal plataforma de alojamiento de proyectos tecnológicos, en los que cualquier persona puede visualizar el código fuente de los mismos}
}

% Digital Twins
\newglossaryentry{Simulador}{
    name=Simulador,
    description= {Especializado en el entorno digital de un sistema, trata de simplificar la complejidad intrínseca del entorno físico obviando los modelos con los que se representarían las características que requerirían una constante comunicación entre ambos \cite{ThreatsDT}}
}

\newglossaryentry{Huella Digital}{
    name=Huella Digital,
    description= {Conjunto de consecuencias reflejadas en el entorno digital del usuario a raíz de sus acciones. Esta información suele quedar plasmada en forma de datos rastreables que permiten reconstruir el modo de proceder del usuario}
}

\newglossaryentry{Avatar Digital}{
    name=Avatar Digital,
    description= {Representación visual del usuario en un determinado entorno digital}
}

% Aplicaciones
\newglossaryentry{contratos inteligentes}{
    name=contratos inteligentes,
    description={Programa informático almacenado en la Blockchain que es capaz de efectuar acciones (ejecutarse) cuando se den las condiciones necesarias}
}

\newglossaryentry{2FA}{
    name=Autenticación en dos pasos (2FA),
    description={Verificación realizada tras la tradicional comprobación del nombre de usuario y la contraseña asociada a la cuenta con la que se quiere iniciar sesión. Con ella, se requiere de una acción adicional desde el dispositivo configurado (como introducir un código autogenerado)}
}

% PoC
\newglossaryentry{entorno de desarrollo}{
    name=entorno de desarrollo,
    description={Conjunto de herramientas técnicas integradas en una misma de interfaz en la que cada una de ellas cumple una función determinada durante la creación de recursos informáticos, principalmente software. El principal cometido de este es facilitar la labor del programador}
}

\newglossaryentry{OpenID}{
    name=OpenID,
    description={Este estándar permite a los usuarios identificarse en los diferentes sitios web con una única cuenta para así evitar la creación de multitud de ellas. Un claro ejemplo de ello es el inicio de sesión mediante una cuenta de Google}
}